\section{Introduction}
  Minimum spanning tree algorithm is an important technique in many fields, which has been researched for a long time. It was first proposed by Otakar Boruvka in 1926\cite{zbMATH02560699} and has been widely used in image segmentation\cite{An2000A, Xu19972D}, classification\cite{Juszczak2009Minimum}, pattern recognition\cite{Zhong2010A}, clustering\cite{Zahn1971Graph, Xu2002Clustering} and machine learning. As an important data mining technique, clustering analysis is the process of dividing a set of objects into non-overlapping subsets, which aims to maximize the inter-cluster similarity and minimize the intra-cluster similarity. Due to its wide application in data mining and other fields, different techniques of clustering have been paid attention over the past decades, such as spectral clustering\cite{Chang2008Robust}, hierarchical clustering\cite{Zhong2011Minimum}, distance and density based clustering\cite{G2016Distance}. 

  The research on MST-based clustering has a long history, which was first proposed by Zahn in 1971\cite{Zahn1971Graph}. Given a weighted and undirected graph, a MST is a connected subgraph that spans over all the vertices without cycles and the total weight of the tree is minimized. Basic idea of MST-based clustering is to find the inconsistent edges of the MST and then remove them. Compared to other clustering methods, MST-based clustering can separate the data sets with different shapes and sizes do not depend on the shape of clusters for the assumptions that data points are grouped around centers or separated by regular geometric curve is not necessary. Traditional MST-based clustering algorithms usually use the distance between two data points as the edge weight, and delete the longest edge to get two clusters. Recent research on MST-based clustering focus more on the efficiency of the algorithm\cite{Jothi2015Fast, Jothi2017Fast}.

  Image segmentation is a fundamental problem in computer vision, which aims at cutting an image into several disjoint subsets such that each subset is a meaningful object of interest in the image. Because of the importance of image segmentation in computer vision, many different segmentation schemes are proposed over the last decades. Among them, the minimal spanning tree based segmentation is substantially related to the graph based clustering\cite{7080012}. The clustering or grouping of image pixels are performed on the minimal spanning tree. The connection of graph vertices satisfies the minimal sum on the defined edge weights, and the partition of a graph is achieved by removing edges to form different sub-graphs.

  Among the previous MST-clustering techniques, many successful ones benefit from improving the quality of clustering. MST-inspired clustering can apply to the outlier detection technique\cite{Wang2012A}. At the same time, the quality of MST-based clustering algorithm is often improved by removing outliers\cite{Wang2013Enhancing}. Besides, neighborhood density estimation is an important technique in the MST-based clustering\cite{Luo2010A}. There are many problems with the traditional MST-based clustering. One of the most shortcomings is that they can't find the different density clusters. To overcome these limitations and get a better result of some special data sets, we present a new method, called scaled-MST, which can apply to finding the different density clusters and clusters with irregular boundaries. Application of edge scaling make good contribution to our algorithms. Furthermore, according to the graph theoretical methods for image processing, we apply our MST-based clustering method to the image segmentation and integration. 

  Our contribution focus on four aspects. Firstly, we propose a new MST method, called scaled-MST algorithm, which works by using scaled edge weights to replace the Euclidean distance between two points. Secondly, we improve the traditional MST-based clustering by deleting the points which have been clustered such that the data space could be reduced largely. Thirdly, we test our method on high dimension(10000) image data sets and then apply to image segmentation. Finally, we integrate five pictures of the image data and improve the efficiency of high dimensional data sets. 

  The rest of paper is organized as follows: In section 2, Some related work on clustering techniques is reviewed. Then we present our method in sec3. Performance evaluation will be given of the comparison method and our method for the sake of the completeness and illustration. So we evaluate our algorithms with respect to other methods. Finally, we make a conclusion and future work in sec5.
