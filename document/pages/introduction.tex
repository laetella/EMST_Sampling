\section{Introduction}
Given a set S of N data points in Euclidean space, the minimum spanning tree (MST) is a spanning tree of a connected, edge-weighted graph with the minimum possible total edge weight. It is a basic data structure in computational geometry. It was first proposed by Otakar Boruvka in 1926\cite{zbMATH02560699}. Minimum spanning tree algorithm has an important application in a wide range of fields, which has been researched for a long time. In two traditional algorithms, Kruscal\cite{Kruskal1956On} and Prim \cite{Prim2013Shortest}, the complexity of the MST computation is $O(n^2)$. However, with the increase of data size in today's world, traditional algorithms are low efficient. Due to its wide applications in a great deal of useful fields, more and more efficient algorithms are proposed in recent years. March et al proposed a Dual Tree Boruvka algorithm, which can improve the efficiency of the algorithms in some extent.

EMST algorithm has been widely used in image segmentation\cite{An2000A, Xu19972D}, classification\cite{Juszczak2009Minimum}, pattern recognition\cite{Zhong2010A}, clustering\cite{Zahn1971Graph, Xu2002Clustering} and machine learning.

We present a new two stage algorithm for efficient construction of MST. Firstly, we sampling from the data sets.
 
The rest of paper is organized as follows: In section 2, Some related work on MST is reviewed. Then we present our method in sec3. Performance evaluation will be given of the comparison method and our method in section 4. Finally, we make a conclusion and future work in sec5. 
 
