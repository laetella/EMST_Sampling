\section{Related work}
  There are two main categories related to the proposed method: MST-based clustering and image segmentation and integration. 
	\subsection{MST-based clustering}
	  MST-based clustering was first proposed by Zahn in 1971\cite{Zahn1971Graph}. The inherent separation was used in this paper to emphasize that the separation rely solely on inter-point distances within the data sets. The inherent relationship between MST and clusters makes the image segmentation performed in an implicit way. However, it is weak in processing the clusters with large variation. 

	  Given a set of N data points and a distance measure defined upon them,modern MST-based clustering algorithms usually begin by constructing an MST. The time complexity of popular MST algorithms, such as the Prim's algorithm, is $O(N^2)$\cite{Cormen2009Introduction}. This time factor limits the application of MST-based clustering methods to massive data sets. Several studies have been done to these problems, which lead to better MST construction efficiency.

	  There have been many efforts in developing new algorithms for MST-based clustering. Many researchers focus more on the efficiency of the construction of MST and have found fast and good algorithms to fit different data sets. Among these efficient algorithms, parallel algorithm is the best method to improve the efficiency\cite{Chong2015An}. However, the parallel random access machine(PRAM) model of computation is needed for the MST study. 

      A kind of efficient MST-inspired clustering algorithm works by finding a local density factor for each data point during the construction of an MST\cite{Wang2013Enhancing}. The application of iDistance significantly reduce the computation time of LOF. 
	  
	  Relying on nothing more than a compressed quadtree data structure to compute approximate EMSTs, the algorithm in \cite{Arya2016A} eliminates the exponential $\epsilon$ dependencies to reduce the time consumption. This algorithm achieves its efficiency by being sloppier in approximating bichromatic closest pairs between well-separated pairs in situations where there is significant EMST weight in the vicinity of the well-separated pair could be infered. 
	  
	  By using a centroid based nearest neighbor rule, a MST-based clustering algorithm is introduced with generating a sparse Local Neighborhood Graph(LNG) and then constructing the approximate MST from LNG in \cite{Jothi2015Fast}. Experimental results demonstrate that the computational time has been reduced significantly by maintaining the quality of the clusters obtained from the MST. 

	  R.Jothi et al proposed two efficient algorithms namely partition based near neighbor graph using Bi-means partitioning (BNNG) and partition based near neighbor graph using K-means partitioning (KNNG) to obtain MST in the context of clustering in a less than quadratic time\cite{Jothi2017Fast}. A novel centroid based nearest neighbor rule is presented in this paper and the experiment results reveal the good performance of the algorithm. 

	  We can use divide-and-conquer scheme to produce an approximate MST\cite{fast_mst_zhong}. There are two stages in this algorithm, divide-and-conquer stage and refinement stage. It is efficiency comparing with other MST algorithms. And the experiment demonstrate that the performance is good on clustering. 

	  Hierarchical and Density based ways are implemented for constructing Minimum Spanning Tree in \cite{Pappula2017An}. The MST can be divided into two segments. The recommended method improves the efficiency of clustering result, where the data is distributed in different shapes and density; it leads to better clustering efficiency. This approach presents a clustering algorithm that is inspired by MST. In this algorithm, a new method for construction which reduces the computational complexity compared with traditional MST construction methods.

	\subsection{image segmentation and integration}
	  Image segmentation is a fundamental problem in computer vision, which has been researched for many years. There are many ways for image segmentation. These methods are categorized into five classes under a uniform notation: the minimal spanning tree based methods, graph cut based methods, the shortest path based methods and the other methods that do not belong to any of these classes\cite{PENG20131020}. A number of research has been focused on these methods. Among them, the minimal spanning tree based segmentation is substantially related to the graph based clustering\cite{7080012}. The clustering or grouping of image pixels are performed on the minimal spanning tree. The connection of graph vertices satisfies the minimal sum on the defined edge weights, and the partition of a graph is achieved by removing edges to form different sub-graphs. 
