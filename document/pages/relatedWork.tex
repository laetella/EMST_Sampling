\section{Related work}
	There are three main categories related to our proposed method: nearest neighbors, sampling and Approxiamate minimum spanning trees. 
	\subsection{Cover tree}
		The main time cost on the construction of MST is the searching of nearest neighbors of data points. Many attention has been paid to improve the efficiency of nearest neighbors. 
		
 		To sampling from the data sets more efficient, wo exploit some useful index methods. 
		In our method, we employ the fast k nearest neighbors: cover tree. 
	\subsection{Sampling}
		There are many sampling methods. Sampling can be applied on many fileds. 
		Center-based sampling method was introduced by Rahnamayan\cite{Rahnamayan2009}. 
		We can learn more about sampling and evaluation in the book published on 2016\cite{Lance2016}. 
		Titsias et.al proposed a new sampling method based on auxiliary gradient\cite{Titsias2018}. They introduce a new family of Markov chain Monte Carlo samplers that combine auxiliary variables, Gibbs sampling and Taylor expansions of the target density. 
	\subsection{AMST}

		Being an important algorithm for computational geometry, MST has been researched for a long time. The studies on MST starts with Boruvka's algorithm \cite{zbMATH02560699}, also called Sollin's algorithm in  parallel computing literature. As in Kruscal's algorithm, it computes MST by adding the minimum weight edge to the minimum spanning forest. At each step, Boruvka's algorithm find all minimum edges incident with each component. This process continues until just one component existing in the graph. Two traditional MST algorithms are Kruscal and Prim, respectively proposed in 1956\cite{Kruskal1956On} and 1957 \cite{Prim2013Shortest}. The famous Prim's algorithm compute MST by selecting an initial point randomly as a tree, and then compute the distance between the start and the other ponits, adding the point which has minimum distance comparing to the others to the tree. Then repeatedly add the nearest point until all the points are in the tree. In Kruscal's algorithm, the edges are sorted by their weights in a non-decreasing order. The minimum edge is adding to the tree if there is no cycle in the tree when adding it to the tree. The time complexity of these two traditional algorithms is $O(E\log V)$. 

		Using the Euclidean distance to compute MST, standard Prim's algorithm requires quardratic running time. In order to improve the efficiency of the MST algorithm, in 1978, Bently and Friedman put forward a new kind of method to speed up the search of nearest point in Prim's algorithm by using kd-tree data structure\cite{Bentley1975Fast}. Then in 1985, Preparata et al. applied an adaptive analysis by giving a lower bound for EMST problem of $O(N\log N)$, which has been the tightest known lower bound\cite{Preparata1985Computational}. Well-separated pair decomposition is a new structure which can greatly improve the performance of MST construction. WSPD was proposed in 1993 by Callahan and Kosaraju\cite{Callahan93fasteralgorithms}. The application of WSPD in computation of MST to compute neighbors of components was then been researched in 2000 by Narasimhan and Zachariasen\cite{Narasimhan00experimentswith}. Unfortunately, the time complexity of MST computation on high-dimensional data sets grows exponentially and is often very large in practice. To overcome this limitation, March et al put forward a new Dual Tree Boruvka algorithm for efficient computation of MST by using cover tree data structure\cite{March2010Fast}. Experiments on large scale astronomical data sets show the good performance of their methods.
