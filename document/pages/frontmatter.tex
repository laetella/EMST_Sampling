
\begin{frontmatter}
	\title{An efficient MST-based clustering algotrithm based on sampling}
	% \author{Jia Li, Xiaochun Wang}
	% \address{laetella827@stu.xjtu.edu.cn, xiaocchunwang@xjtu.edu.cn}
	\begin{abstract}
		Euclidean Minimum Spanning Tree algorithm is an important method in many fields, and abundant developments have made it a significant branch of mordern graph-based algorithms. However, for today's real-world high dimensionality and huge size of data sets, the requirements of high execution efficiency and high accuracy of clustering result cannot be met at the same time. To address this issue, this paper presents a new method for MST using sampling. In the first step, we sampling from the data sets by using nearest neighbors to reduce the scale of data sets. Then we construct an MST of sampling data sets. Then we insert the remainned data to the MST. Experiment results demonstrate the efficacy of our method on high dimensional data sets. 
	\end{abstract}
	
	\begin{keyword}
		Sampling\sep AMST \sep  minimum spanning tree 
	\end{keyword}
\end{frontmatter}
